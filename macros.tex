% Utils
% \newcommand{\todo}[1]{\textcolor{red}{\MakeUppercase{#1}}\\}
\newcommand{\tocite}[1]{ \textcolor{red}{[#1]} } 
\newcommand{\addlink}[2]{\href{#1}{#2}\footnote{\url{#1}}}
\def\code#1{\texttt{#1}}

\newcommand{\censor}[1]{#1}

% Standard algos
\newcommand{\neat}{NEAT}
\newcommand{\hyperneat}{HyperNEAT}
% \newcommand{\es}{evolution strategies}
\newcommand{\snes}{SNES}
\newcommand{\xnes}{XNES}
\newcommand{\enes}{eNES} 
\newcommand{\cmaes}{CMA-ES}
\newcommand{\sepcmaes}{sep-CMA-ES}
\newcommand{\openaies}{OpenAI ES}
\newcommand{\canonical}{Canonical ES}
\newcommand{\direct}{direct encoding}
\newcommand{\ars}{Augmented Random Search}


% My algos
%% GENE
% \newcommand{\gene}{GENE}
\newcommand{\coordsnet}{GENE}
\newcommand{\xdgene}{XD-GENE}
\newcommand{\ltwogene}{L2-GENE}
\newcommand{\grngene}{tag-GENE}
\newcommand{\pltwogene}{pL2-GENE}
\newcommand{\coordinates}{\textit{coordinates}}
\newcommand{\distfunc}{\textit{distance function}}

%% ES
% \newcommand{\dqnes}{DQNES}
\newcommand{\elitES}{ElitES}
% \newcommand{\mujoco}{\textsc{MuJoCo}}
\newcommand{\customes}{Custom ES}

\newcommand{\berl}{\code{BERL}}
\newcommand{\cambrian}{\code{Cambrian.jl} }

% Papers
\newcommand{\openaipaper}{\cite{salimansEvolutionStrategiesScalable2017}}
\newcommand{\canonicalpaper}{\cite{chrabaszczBackBasicsBenchmarking2018} }
\newcommand{\nespaper}{\cite{wierstraNaturalEvolutionStrategies2014}}
\newcommand{\aleguidelines}{\cite{machadoRevisitingArcadeLearning2017}}



%%% Text
\newcommand{\ie}{\emph{i.e.}}
\newcommand{\eg}{\emph{e.g.}}
\newcommand{\wrt}{\emph{w.r.t.}}
\newcommand{\etc}{\emph{e.t.c.}}
\newcommand{\onemax}{\textsc{OneMax}}
\newcommand{\leadingones}{\textsc{LeadingOnes}}
\newcommand{\procgen}{\textsc{ProcGen}}
\newcommand{\cartpole}{\textsc{CartPole}}
\newcommand{\acrobot}{\textsc{Acrobot}}
\newcommand{\pendulum}{\textsc{Pendulum}}
\newcommand{\mujoco}{\textsc{MuJoCo}}
\newcommand{\qm}[1]{``#1''}
% \newcommand{\note}[1]{{\color{blue}#1}}
\newcommand{\honemax}{H2}
\newcommand{\hleadingones}{H5}
\newcommand{\red}[1]{{\color{red}#1}}
% \newcommand{\lucie}{{\color{magenta}\texttt{LUCIE}}}

%%% Common Maths
\newcommand{\N}{\ensuremath{\mathbb{N}}}
\newcommand{\Z}{\ensuremath{\mathbb{Z}}}
\newcommand{\R}{\ensuremath{\mathbb{R}}}
\newcommand{\lnof}[1]{\ensuremath{\ln\left(#1\right)}}
\newcommand{\expof}[1]{\ensuremath{\exp\left(#1\right)}}
\newcommand{\powerof}[2]{\ensuremath{\left(#1\right)^{#2}}}
\newcommand{\set}[1]{\ensuremath{\left\{#1\right\}}}
\newcommand{\expect}{\ensuremath{\mathbb{E}}}
\newcommand{\expectof}[1]{\ensuremath{\expect{}\left(#1\right)}}
\newcommand{\proba}{\ensuremath{\mathbb{P}}}
\newcommand{\probaof}[1]{\ensuremath{\proba{}\left(#1\right)}}
\newcommand{\probaofgiven}[2]{\ensuremath{\proba{}\left(#1\;\middle|\;#2\right)}}
\newcommand{\union}[2]{\ensuremath{\bigcup\limits_{#1}^{#2}}}
\newcommand{\sumc}[2]{\ensuremath{\sum_{#1}^{#2}}} % condensed notation for sum
\newcommand{\intc}[2]{\ensuremath{\int_{#1}^{#2}}} % condensed notation for integral
\newcommand{\indic}[1]{\ensuremath{\boldsymbol{1}\left(#1\right)}} % indicatrice
%\newcommand{\intrange}[2]{\ensuremath{\left[#1,#2\right]}}
\newcommand{\intrange}[2]{\ensuremath{\left\{#1,\dots,#2\right\}}}
\newcommand{\eqdef}{\ensuremath{\stackrel{\text{def}}{=}}}
\newcommand{\bigo}{\ensuremath{\mathcal{O}}}
\newcommand{\bigoof}[1]{\ensuremath{\bigo{}\left(#1\right)}}
% \DeclarePairedDelimiter{\ceil}{\lceil}{\rceil}
% \DeclareMathOperator\erf{erf}
% \DeclareMathOperator*{\argmax}{arg\,max}
% \DeclareMathOperator*{\argmin}{arg\,min}

%%% Our Maths

% Constants
\newcommand{\csta}{\ensuremath{4}}
\newcommand{\cstatimestwo}{\ensuremath{8}}
\newcommand{\cstaminusone}{\ensuremath{3}}
\newcommand{\cstaminustwo}{\ensuremath{2}}
\newcommand{\cstb}{\ensuremath{2}}
\newcommand{\cst}[1]{\ensuremath{C_{#1}}}
\newcommand{\gammamax}{\ensuremath{0.57}}
\newcommand{\cstlemmathree}{\ensuremath{\tilde{C}}}
\newcommand{\npopmin}{\ensuremath{2}} % min total population size
\newcommand{\tmin}{\ensuremath{2}} % min number of steps
\newcommand{\deltamax}{\ensuremath{0.5}} % max value of delta

\newcommand{\npop}{\ensuremath{n}} % total population size
\newcommand{\nresampling}{\ensuremath{n_{\text{RS}}}}
\newcommand{\scalingfactor}{\ensuremath{\alpha}}
\newcommand{\nbits}{\ensuremath{N}} % Number of bits in OneMax and LeadingOnes
\newcommand{\budget}{\ensuremath{B}} % Max number of eval per generation for LUCBEA
\newcommand{\samplingstrategy}{\ensuremath{\mathcal{S}}}
\newcommand{\pbcomplexity}[2]{\ensuremath{H^{#1,#2}}}% pb complexity at gen #1 with factor #2
\newcommand{\meansymbol}[1]{\ensuremath{f}} % mean symbol
\newcommand{\mean}[1]{\ensuremath{\meansymbol{}_{#1}}}% mean of ind #1
\newcommand{\empiricalmean}[3]{\ensuremath{\hat{\meansymbol{}}_{#1}^{#2,#3}}}% empirical mean of ind #1 at gen #2 at time #3
\newcommand{\ttotal}{\ensuremath{N_{\text{non-term}}}}% total number of steps
\newcommand{\tremain}{\ensuremath{N_{\text{remain}}}}% number of remaining steps
\newcommand{\samplessymbol}{\ensuremath{u}}
%\newcommand{\samplesgensymbol}{\ensuremath{v}}
\newcommand{\nones}[1]{\ensuremath{D^{#1}}}% number of ones at generation #1
\newcommand{\opt}{\ensuremath{\text{opt}}}
\newcommand{\pes}{\ensuremath{\text{pes}}}
\newcommand{\xpesrv}[1]{\ensuremath{X^{\pes{}}_{#1}}}% rv X pessimal in leadingones with #1 leading ones
\newcommand{\xoptrv}[1]{\ensuremath{X^{\opt{}}_{#1}}}% rv X optimal in leadingones with #1 leading ones
\newcommand{\xpes}[1]{\ensuremath{x^{\pes{}}_{#1}}}% X pessimal in leadingones with #1 leading ones
\newcommand{\xopt}[1]{\ensuremath{x^{\opt{}}_{#1}}}% X optimal in leadingones with #1 leading ones
\newcommand{\observationsymbol}[1]{\ensuremath{O}}
\newcommand{\observationof}[1]{\ensuremath{\observationsymbol{}\left(#1\right)}}% observation of an ind #1 fitness
\newcommand{\nsamples}[3]{\ensuremath{\samplessymbol{}_{#1}^{#2,#3}}}% TOTAL number of samples of ind #1 at gen #2 at time #3
%\newcommand{\nsamplesgen}[3]{\ensuremath{\samplesgensymbol{}_{#1}^{#2,#3}}}% number of samples of ind #1 collected during gen #2 at step #3
\newcommand{\nsamplesstar}[3]{\ensuremath{\samplessymbol{}_{#1}^{*}(#3)}}% TOTAL number of samples of ind #1 at gen #2 at step #3 for bound to be lesser than max(delta_ind, epsilon/2)
%\newcommand{\nsamplesstargen}[3]{\ensuremath{\samplesgensymbol{}_{#1}^{*}(#2,#3)}}% number of samples of ind #1 collected during step #2 at time #3 for bound to be lesser than max(delta_ind, epsilon/2)
\newcommand{\nsamplesinit}[2]{\ensuremath{U_{#1}^{#2}}}% initial number of samples of ind #1 at gen #2
\newcommand{\nsamplesinitmax}[1]{\ensuremath{U^{#1}}}% max initial number of samples of gen #1
\newcommand{\boundsymbol}{\ensuremath{\beta}}
\newcommand{\bound}[3]{\ensuremath{\boundsymbol{}\left(#1,#3\right)}}% bound after #1 samples at step #3 of ind #2
\newcommand{\boundbarsymbol}{\ensuremath{\bar{\beta}}}
\newcommand{\boundbar}[3]{\ensuremath{\boundbarsymbol{}\left(#1,#3\right)}}% bound after #1 samples at step #3 of ind #2
\newcommand{\hucb}[2]{\ensuremath{h_{*}^{#1,#2}}} % highest UCB at gen #1 step #2
\newcommand{\llcb}[2]{\ensuremath{l_{*}^{#1,#2}}} % lowest LCB at gen #1 step #2
\newcommand{\hucbshort}[2]{\ensuremath{h^{#2}}}
\newcommand{\llcbshort}[2]{\ensuremath{l^{#2}}}
\newcommand{\maxc}[2]{\ensuremath{\left[#1\wedge#2\right]}} % condensed notation for max

% Sets
\newcommand{\individuals}[1]{\ensuremath{\textsc{Ind}^{#1}}} % Set of individuals of gen g
\newcommand{\topset}[1]{\ensuremath{\textsc{Top}^{#1}}} % top mu individuals of gen g
\newcommand{\botset}[1]{\ensuremath{\textsc{Bot}^{#1}}} % non-top mu individuals of gen g
\newcommand{\goodset}[1]{\ensuremath{\textsc{Good}^{#1}}} % epsilon-mu-optimal individuals of gen g
\newcommand{\badset}[1]{\ensuremath{\textsc{Bad}^{#1}}} % non-epsilon-mu-optimal individuals of gen g
\newcommand{\highset}[2]{\ensuremath{\textsc{High}^{#1,#2}}} % estimated top mu individuals at step t of gen g
\newcommand{\lowset}[2]{\ensuremath{\textsc{Low}^{#1,#2}}} % estimated non-top mu individuals at step t of gen g
\newcommand{\aboveset}[2]{\ensuremath{\textsc{Above}^{#1,#2}}} % middle at step t of gen g
\newcommand{\middleset}[2]{\ensuremath{\textsc{Middle}^{#1,#2}}} % middle at step t of gen g
\newcommand{\belowset}[2]{\ensuremath{\textsc{Below}^{#1,#2}}} % middle at step t of gen g

% Events
\newcommand{\failgen}[1]{\ensuremath{F^{#1}}}% fail gen #1
\newcommand{\failgenstep}[2]{\ensuremath{F^{#1}_{#2}}}% fail gen #1 step #2
\newcommand{\cross}[2]{\ensuremath{\textsc{Cross}^{#1,#2}}}
\newcommand{\crossind}[3]{\ensuremath{\textsc{Cross}_{#1}^{#2,#3}}}
\newcommand{\needy}[3]{\ensuremath{\textsc{Needy}_{#1}^{#2,#3}}}
\newcommand{\terminate}[2]{\ensuremath{\textsc{Term}^{#1,#2}}}


% My macros...
\newcommand*{\SET}[1]  {\ensuremath{\boldsymbol{#1}}}
\newcommand*{\VEC}[1]  {\ensuremath{\boldsymbol{#1}}}
\newcommand*{\MAT}[1]  {\ensuremath{\boldsymbol{#1}}}
\newcommand*{\OP}[1]  {\ensuremath{\text{#1}}}
\newcommand*{\NORM}[1]  {\ensuremath{\left\|#1\right\|}}
\newcommand*{\DPR}[2]  {\ensuremath{\left \langle #1,#2 \right \rangle}}
\newcommand*{\calbf}[1]  {\ensuremath{\boldsymbol{\mathcal{#1}}}}
\newcommand*{\shift}[1]  {\ensuremath{\boldsymbol{#1}}}
\newcommand{\argmax}{\operatornamewithlimits{argmax}}
\newcommand{\argmin}{\operatornamewithlimits{argmin}}
\newcommand{\ud}{\, \text{d}}
\newcommand{\vect}{\text{Vect}}
\newcommand{\sinc}{\text{sinc}}
\newcommand{\esp}{\ensuremath{\mathbb{E}}}
\newcommand{\hilbert}{\ensuremath{\mathcal{H}}}
\newcommand{\fourier}{\ensuremath{\mathcal{F}}}
\newcommand{\sgn}{\text{sgn}}
\newcommand{\intTT}{\int_{-T}^{T}}
\newcommand{\intT}{\int_{-\frac{T}{2}}^{\frac{T}{2}}}
\newcommand{\intinf}{\int_{-\infty}^{+\infty}}
\newcommand{\Sh}{\ensuremath{\boldsymbol{S}}}
\newcommand{\Cpx}{\ensuremath{\mathbb{C}}}
\newcommand{\K}{\ensuremath{\mathbb{K}}}
\newcommand{\reel}{\mathcal{R}}
\newcommand{\imag}{\mathcal{I}}
\newcommand{\cmnr}{c_{m,n}^\reel}
\newcommand{\cmni}{c_{m,n}^\imag}
\newcommand{\cnr}{c_{n}^\reel}
\newcommand{\cni}{c_{n}^\imag}
\newcommand{\LR}{\mathcal{L}_2(\R)}
\newcommand{\tproto}{g}
\newcommand{\rproto}{\check{g}}
\newcommand{\Tproto}{G}
\newcommand{\Rproto}{\check{G}}

%%% TIKZ

% fitness function
%\newcommand{\fit}[1]{{3+sin(54*#1-40)-0.2*#1}}
\newcommand{\fit}[1]{{2-0.1*#1+2.5*exp(-0.3*(#1-3)^2)+1.5*exp(-0.5*(#1-7)^2)}}

% plot a generation of indiv + their fitness
\newcommand{\plotgenf}[4]{% x color in-slide out-slide
\foreach \x in {#1}{%
	\onslide<#3-#4>{\node[ind,fill=#2] at (\x,0){};}
	\onslide<#4-#4>{\node[ind,fill=#2] at (\x,\fit{\x}){};}
}
}

% plot a generation of indiv + their fitness
\newcommand{\plotgen}[4]{% coord color in-slide out-slide
\foreach \x/\y in {#1}{%
	\onslide<#3-#4>{\node[ind,fill=#2] at (\x,0){};}
	\onslide<#4-#4>{\node[ind,fill=#2] at (\x,\y){};}
}
}

% plot a generation but only fitness
\newcommand{\plotgenonlyeval}[4]{% coord color in-slide out-slide
\foreach \x/\y in {#1}{%
	\onslide<#4-#4>{\node[ind,fill=#2] at (\x,\y){};}
}
}

% plot a one indiv and many eval
\newcommand{\plotgenmanyeval}[5]{% x ys color in-slide out-slide
	\onslide<#4-#5>{\node[ind,fill=#3] at (#1,0){};}
	\foreach \y in {#2}{%
		\onslide<#4-#5>{\node[ind,fill=#3] at (#1,\y){};}
	}
}

\newcommand{\squarebracket}[6]{% x y width color sidesize thickness
	\draw[color=#4, #6] (#1-#3,#2+#5) -- (#1-#3,#2) -- (#1+#3,#2) -- (#1+#3,#2+#5);
}

\newcommand{\roundbracket}[7]{% x y width color sidesize thickness rotate
	\begin{scope}[shift={(#1,#2)},rotate=#7]%
		\draw[color=#4,#6] (-#3+#5,0) -- (#3-#5,0);%
		\draw[color=#4,#6] (-#3,+#5) arc (0:90:-#5);%
		\draw[color=#4,#6] (#3,+#5) arc (0:-90:#5);%
	\end{scope}%
}

\def\bw{0.1} % confidence intervals width
\newcommand{\cifit}[4]{% x y beta color
	\draw[ultra thick, color=#4] (#1,#2-#3) -- (#1,#2+#3);
	\draw[ultra thick, color=#4] (#1-\bw,#2-#3) -- (#1+\bw,#2-#3);
	\draw[ultra thick, color=#4] (#1-\bw,#2+#3) -- (#1+\bw,#2+#3);
	\node[circle, draw=none, minimum width=0.2cm, inner sep=0, outer sep=0, fill=white, color=#4] at (#1,#2) {};
}